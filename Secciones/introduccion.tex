\section{Introducción} 
inicio al estudio de un modelo de dispersión elástica de nucleones (y anti-nucleones) el cual se basa en una representación quark-diquark\footnote{El diquark un estado ligado hipotético de dos quarks. El diquark es a menudo tratado como una sola partícula subatómica con la que el tercer quark interactúa a través de la interacción fuerte.} ($qQ$) del nucleón con
elasticidad del pomerón ($\alpha_p$) propuesto por Vladimir Grichine\footnote{En lo que resta del del texto nos limitamos únicamente a colisiones $p(\overline{p})-p$, este es el caso para el cual se tiene una mayor cantidad de datos. No obstante, el modelo \cite{grichine} es v\'alido para cualquier tipo de nucle\'on.} . Este nuevo modelo (que trata de describir datos de dispersión nuclear para el cual 0.005$<|t|<$1 GeV$^2$, región no perturbativa) trata de mejorar el trabajo de la referencia \cite{modeloqQ}  en donde se puede ver una fuerte sobre-estimación en el mínimo de difracción, esto se debe a que el valor de la parte real de la amplitud de dispersión $F(s,t)$ no es lo suficientemente grande \cite{grichine}. Los ajustes presentados en las referencias  \cite{modeloqQ} y \cite{ivan} incorporan dos parámetros libres: un coeficiente de pendiente nuclear y el radio del prot\'on. La nueva función  prototipo $F(s,t)$ presentada por Grichine en la cita \cite{grichine} es capaz de mejorar la descripción de los datos en el m\'inino de difracción incluyendo el nuevo parámetro empírico $\alpha_p$ que aumenta la parte real de $F(s,t)$. Sin embargo, en el siguiente cap\'itulo veremos que a pesar de que el modelo en la referencia \cite{grichine} mejora considerablemente los fits, este tiene sus limitaciones (como sucede con cualquier modelo fenomenológico), por lo que solo puede ser aplicado para datos correspondientes a ciertas energías. Este problema se debe a que la normalización global de $F(s,t)$ est\'a fija y el programa falla en determinar la posición del minino de difracción. Una modificación al modelo, es dejar como parámetro libre la normalización global, en consecuencia, los resultados obtenidos son muy satisfactorios, describiendo todos los datos existentes de sección eficaz diferencial protón-protón y protón-antiprotón en un rango amplio de energías ( 4 GeV a 7000 GeV). En lo que resta de este apartado, daremos a conocer las características del modelo $(qQ)$ con pomer\'on el\'astico.
