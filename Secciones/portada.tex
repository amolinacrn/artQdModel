\runningheads{AIRQUD ~Plugin para QGIS}{Modelo basico de la dispersion de los contaminantes }

\title{\fontsize{17}{11pt}\selectfont Sección eficaz diferencial elástica de protones en una representa\-ción quark-diquark con pomer\'on elástico y normalización global}

\author{\fontsize{9.0}{10.5pt}\selectfont Carlos \'Avila Bernal{\small$^1$}, Alejandro Molina Cer\'on{\small $^2$}\\ \vskip -1.1em
  {\small $^1$}Departamento de f\'isica, Universidad de los Andes \\ \vskip -1.1em
   {\small $^2$}Departamento de f\'isica, Universidad de Nari\~no \\ \vskip -1.1em
Febrero 28 2018 }

\address{ }
\begin{abstract}
Se presenta el estudio de un modelo de dispersión elástica de nucleones (y anti-nucleones) basado  en una representación quark-diquark $(qQ)$  del nucleón con pomerón elástico. Este modelo aumenta la parte real de  la amplitud de dispersión mejorando su descripción en el mínimo de difracción. Las predicciones del modelo se comparan con los datos experimentales disponibles para cada una de las secciones eficaces diferenciales elásticas  de los nucleones en  un rango de energía entre 4.26 GeV hasta 7 TeV. Esta parametrización no describe correctamente los  datos experimentales para ciertas energías en colisiones prot\'on-prot\'on (y antiprot\'on-prot\'on). Se incluye un estudio que trata de mejorar el modelo de ajuste a  los datos existentes, presentando buenos resultados.
\end{abstract}
\keywords{pomer\'on elástico, normalización global, dispersión elástica, amplitud de dispersión.}
\begin{engabstract}
{\sffamily\bfseries\selectfont Sección eficaz diferencial elástica de protones en una representación quark-diquark con pomer\'on elástico y normalización global}
 \vskip 0.8em
We present a study of the nucleon-nucleon (an nucleon-antinucleon) elastic differential cross section  based on a representation quark-diquark (qQ) of the nucleon with elastic pomeron. This model increases the real part of the scattering amplitude improving the description of the diffraction minimum. The predictions of the model are compared to the available experimental data within the range of 4.26 GeV through 7 TeV. This parametrization does not describe correctly the experimental data of proton-proton (proton-antiproton) collissions for several energies. We include a study aiming to improve the model description to existing data, obtaining good results.
\end{engabstract}
\keywordseng{Extensión; Qgis; GRASS; Interpolación; RST; IDW}
\maketitle
